%!TEX root = ../main.tex

\chapter{Thermodynamics with Gradient Variables}

In this chapter we present the basic relations of continuum
thermodynamics for solid material behavior with ``nonlocal'', i.e. 
gradient, effects. Constitutive relations are established for the most 
general situation of non-isothermal behavior.

\section{Literature overview}

{\scshape Truesdell}~\cite{Truesdell1968}, {\scshape Noll}~\cite{Noll1958},
 {\scshape Truesdell \&
Noll}~\cite{Truesdell1965}, and {\scshape Coleman}~\cite{Coleman1964}, 
text-books by {\scshape Lemaitre \&
Chaboche}~\cite{Lemaitre1990a} and {\scshape Maugin}~\cite{Maugin1992}.

--------------------------------------------------------------

\subsection{Modeling school A?}
sdffsdfdsfdsfsd
fsd
fsd
fsd
f
sdf
sd

fsdfds

\subsection{Modeling school B?}
fsdfsd
fsd
fs
dfs

sfsdfs


sfsdfsd
\section{Thermodynamic concepts in the presence of gradient variables}
gfdgfdg
dfg
dg
df

gdffgd
\subsection{Thermodynamic system}

\textcolor{red}{Some introduction to thermodynamic system. How we use it.}
\begin{definition}[Closed thermodynamic system]

The standard definition of a {\em closed thermodynamic system} is a macroscopic
volume in space that is closed in the sense that it can not exchange matter with the
surroundings. It can, however, exchange energy in various forms with the 
surroundings through the boundary of the volume.\footnote{This 
definition excludes ``growth'' whereby the mass balance law involves a source term.}
\end{definition}

When establishing balance equations, it is convenient to consider an arbitrary part
of an existing finite body whose boundary is subjected to given boundary 
conditions. In other words, we consider the ``cut-out'' amount of matter that 
occupies the spatial region $\body$ with boundary  
$\boundary$ (with outward unit normal $\normal$, as shown in Figure
\ref{fig:vol_I_ch02_fig1}. Here, we consider only small deformations, whereby
the mass density $\rho$ is taken as a material parameter
(rather than a field variable) that is unaffected by the deformation. 
This ``body'' is acted upon by two types of forces, macroforces and microforces.

\begin{definition}[Macroforces]
\index{Definitions!Macroforces}
Macroforces represents ``local'' or ``mechanical'' forces in continuum models
 that are conjugated with the macro-displacement $\u$ of the body. 
The macroforces are made up from two types of forces. A body force ${\bm b}$ 
(force per unit volume) acting in 
the interior of $\body$, and surface tractions $\tract$  
acting on $\boundary$ from the surrounding matter 
(from which the considered body is cut-out from).
\end{definition}

\begin{definition}[Microforces]
\index{Definitions!Microforces}
Microforces represents ``non-local'' or ``gradient'' forces in continuum models
that are conjugated to the ``micro-displacement'' $u^\micro=k$,
which is identical to the internal variable. 
A volume-specific ``micro-force'' $b^\micro$ is acting in the interior 
of $\body$, and a surface ``micro-traction'' $\microtract$
is acting on $\boundary$ from the surrounding matter
\footnote{To simplify notation, we assume in this chapter that $b^\micro$ 
and $\microtract$ are scalar 
quantities}
\end{definition}

\subsubsection{Heat sources}
A volume-specific heat source $r$ (power per unit volume) acts in the interior
of $\body$, and the thermal power $q_\subn$ is supplied
via $\boundary$. 
It is assumed that there is no micro-source of heat present.

\section{Balance laws}

\subsection{Momentum balance}
\textcolor{red}{$\mathcal{P}$ has no bold in this font}

The total momentum $\momum$ of the body $\body$ in Figure
\ref{fig:vol_I_ch02_fig1} is given as
%--------------------------------------------------------------------------------
\begin{equation}
  \momum = \vint{\rho \dot{\bm {u}}},
  \label{eq:vol_I_ch02_eq3}
\end{equation}
%------------------------------------------------------------------------------
whereas the resultant $\force_\ext$ of the externally
applied mechanical loads on $\mathcal{B}$ is
%-----------------------------------------------------------------------------
\begin{equation}
  \force_\ext =\vint{\bm{b}} + \sint{\tract}.
    \label{eq:vol_I_ch02_eq4}
\end{equation}
The global format of the momentum balance reads
\begin{equation}
  \dot{\momum}= \force_\ext.
     \label{eq:vol_I_ch02_eq5}
\end{equation}

\begin{theorem} \label{thm:equilibrium}
\index{Theorems!Equilibrium}
The equations of motion can be written as 
 $\rho \uddot - \stress \cdot \grad = \bm{b}$. 
\end{theorem}
\begin{proof}
We may use the relation $\bm{t}_\norm = \stress \cdot \normal$
between the (symmetric) stress tensor $\stress$ and the total traction
$\tract$ along the boundary $\boundary$ together with
Green's theorem to transform the boundary integral in
(\ref{eq:vol_I_ch02_eq4}) into a volume integral, which gives
\begin{equation}
  \force_\ext = \vint{\bm{b} + \stress \cdot \grad}
     \label{eq:vol_I_ch02_eq6}
\end{equation}
Upon inserting \eqref{eq:vol_I_ch02_eq3} and
\eqref{eq:vol_I_ch02_eq6} into \eqref{eq:vol_I_ch02_eq5}, we obtain
\begin{equation}
   \vint{\rho \uddot} = \vint{ \bm{b} + \stress \cdot \grad}
  \label{eq:vol_I_ch02_eq7}
\end{equation}
Since the volume $\body$ is arbitrary the integrands must be equal pointwise
which is exactly \ref{thm:equilibrium}.
\end{proof}
\begin{remark}
When inertia forces are ignored ($\rho \uddot = \bm{0}$), then
\ref{thm:equilibrium} represents the quasistatic equilibrium
equation.
\end{remark}

In a completely analogous fashion, we establish the
resultant $\micforce$ of the externally applied micro-forces on $\body$ as
%-----------------------------------------------------------------------------
\begin{equation}
  \micforce = \vint{b^\micro} + \sint{ \microtract}.
    \label{eq:vol_I_ch02_eq4a}
\end{equation}
%----------------------------------------------------------------------
Clearly, the global format of micro-force equilibrium is
%-----------------------------------------------------------------------------
\begin{equation}
 \micforce = 0.
    \label{eq:vol_I_ch02_eq4b}
\end{equation}

\begin{theorem} \label{thm:micro_equilibrium}
\index{Theorems!Micro-equilibrium}
The equation of micro-equilibrium can be written as
   $- \microstress \cdot \grad = b^\micro$
\end{theorem}
\begin{proof}
The proof is done in the same way as theorem \ref{thm:equilibrium}.
\end{proof}



\subsection{Energy balance}
We first define the standard forms of {\em mechanical} and {\em thermal power} supply.
\begin{definition}[Mechanical power]
\index{Definitions!Mechanical power}
\begin{equation}
  \mechpow \defeq
  \vint{{\bm b} \cdot \udot} +
  \sint{\tract \cdot \udot},
  \label{eq:vol_I_ch02_eq16}
\end{equation}
\end{definition}

\begin{definition}[Thermal power]
\index{Definitions!Thermal power}
\begin{equation}
  \thermpow \defeq
  \vint{r} + \sint{q_n},
\label{eq:vol_I_ch02_eq17}
\end{equation}
\end{definition}
We also introduce the conecpt of micropower.
\begin{definition}[Micropower]
The micropower is the power supply from the microforces
\index{Definitions!Micropower}
\begin{equation}
  \micpow \defeq
  \vint{ b^\micro \dot{k} } +
  \sint{ \microtract \dot{k} }.
\label{eq:vol_I_ch02_eq16a}
\end{equation}
\end{definition}




We first introduce the concepts of kinetic and total internal energy.
\begin{definition}[Kinetic energy]\label{def:KE}
\index{Definitions!Kinetic energy}
The kinetic energy ${\mathcal K}$ is defined as
\begin{equation}
  \mathcal{K} \defeq \frac{1}{2} \vint{ \rho | \dot{\bm {u}} | ^2}
\label{eq:vol_I_ch02_eq14}
\end{equation}
\end{definition}

\begin{definition}[Total internal energy]\label{def:IE}
\index{Definitions!Total internal energy}
The total internal energy $\mathcal{E}$ is defined as
\begin{equation}
   {\mathcal{E}} \defeq \vint{\rho e}
\label{eq:vol_I_ch02_eq15}
\end{equation}
where $e$ is the internal energy per unit mass.
\end{definition}


In standard fashion, energy is supplied to the system in the form of
mechanical power $\mechpow$ (from the mechanical loads
${\bm b}$ and ${\bm t}$) and in the form of {\em thermal power} 
$\thermpow$ (from the thermal ``loads'' $r$ and $q$).
However, energy is also supplied from ``micro-power''
$\micpow$ (generated by $b^\micro$ and $\microtract$).
Part of the supplied energy may be converted into kinetic
energy ${\mathcal K}$, which is manifested by macroscopic motion of the
body ($\udot \neq \bm{0}$); hence, it is assumed that the micro-motion 
$\dot{k}$ does not constitute any part of ${\mathcal K}$ since it is not
 associated with any ``micro-inertia''. 
The remaining part of the energy is
stored as internal energy ${\mathcal E}$, which has to be
parameterized suitably in terms of thermodynamically independent variables.
How the total energy can change is governed by the first law of thermodynamics.
\begin{physlaw}[First law of thermodynacmics] 
\index{Physical laws!First law of thermodynamics}
This law concern the conservation 
of energy in a thermodynamic systems. It says that the change in the total 
energy of a thermodynamic is equal to the external power flowing into the 
system... \textcolor{red}{FIX}
\begin{equation}
  \mathcal{\dot{E}} + \mathcal{\dot{K}} = \mechpow + \micpow + \thermpow
\label{eq:vol_I_ch02_eq18}
\end{equation}
\end{physlaw}

The precise way in which the total energy ${\mathcal{E}}+{\mathcal K}$ may
change in a given thermodynamic process is discussed next.


\begin{theorem}[The local energy equation]
\label{thm:local_energy_eq}
\index{Thorem!The local energy equation}
The energy in the system satisfies
\begin{equation}
  \rho \dot{e} =
  \stress : \dot{\strain} + \microstress \cdot \gdot + r - {\bm h}\cdot{\bm \nabla}.
   \label{eq:vol_I_ch02_eq25}
\end{equation}
\end{theorem}
\begin{proof}
We may use the relations $\tract = \stress \cdot \normal$ and 
$\microtract  = \microstress \cdot \normal$
together with Green's theorem to transform the boundary integrals in
(\ref{eq:vol_I_ch02_eq16}) and (\ref{eq:vol_I_ch02_eq16a}) into volume
 integrals, which gives
%-----------------------------------------------------------------------------
\begin{equation}
  \mechpow = 
  \vint{ \bm{b} \cdot \udot} +
  \vint{ [\stress \cdot \udot] \cdot \grad } =
  \vint{ [\bm{b} + \stress \cdot \grad] \cdot \udot} +
  \vint{ \stress : \dot{\strain}}
\label{eq:vol_I_ch02_eq19}
\end{equation}
%--------------------------------------------------------------------------------
%-----------------------------------------------------------------------------
\begin{equation}
  \micpow = 
  \vint{ b^\micro \dot{k} } +
  \vint{ [ \microstress \dot{k}] \cdot \grad } =
  \vint{ [b^\micro + \microstress \cdot \grad]\dot{k} } +
  \vint{ \microstress \cdot \gdot }
  \label{eq:vol_I_ch02_eq19a}
\end{equation}
%--------------------------------------------------------------------------------
Here, we introduced the strain and gradient operators $\strain$ and $\g$

\textcolor{red}{Maybe not new operators introduced in a proof...}
%-----------------------------------------------------------------------------
\begin{subequations}\label{eq:vol_I_ch02_eq119}
\begin{alignat}{2}
    \strain[\u]& \defeq (\u \otimes \grad)^{\mathrm{sym}} \quad 
    \Rightarrow \quad \dot{\strain} & =\strain[\udot],
\label{eq:vol_I_ch02_eq119a} \\
    \g[k] &\defeq k \otimes \grad  = \grad k \quad 
    \Rightarrow \quad \gdot & = \g[\dot{k}].
\label{eq:vol_I_ch02_eq119b}
\end{alignat}
\end{subequations}
%--------------------------------------------------------------------------------
In order to obtain the last expression in (\ref{eq:vol_I_ch02_eq19}), 
we used that $\stress$ is
symmetrical, which implies that
 $\stress: \left[ \udot \otimes \grad \right] = \stress:\dot{\strain}$.

Now, inserting the equations of equilibrium in (\ref{eq:vol_I_ch02_eq7}) 
and micro-equilibrium in (\ref{eq:vol_I_ch02_eq7a})
into (\ref{eq:vol_I_ch02_eq19}) and (\ref{eq:vol_I_ch02_eq19a}), we obtain
%-----------------------------------------------------------------------------
\begin{equation}
  \mechpow = 
  \vint{ \uddot \cdot \rho \udot + \stress : \dot{\strain}} =
  \vint{ \ddt \left[\frac{1}{2} \rho|\udot|^2 \right] + \stress : \dot{\strain} }
\label{eq:vol_I_ch02_eq21}
\end{equation}
%--------------------------------------------------------------------------------
%-----------------------------------------------------------------------------
\begin{equation}
  \micpow = 
  \vint{\microstress \cdot \dot{\bm g}}
\label{eq:vol_I_ch02_eq21a}
\end{equation}
%--------------------------------------------------------------------------------

Likewise, we may introduce the heat flux vector $\bm h$ and use the
relation $q_\subn = -{\bm h}\cdot \normal $ to transform the
boundary integral in (\ref{eq:vol_I_ch02_eq17}) into a volume
integral. We then obtain 
%-----------------------------------------------------------------------------
\begin{equation}
  \thermpow = \vint{ \mathcal Q}  \quad
  \mathrm{with}
  \quad {\mathcal Q} \defeq - \grad \cdot \bm{h}
  %\label{eq:203.13b}
       \label{eq:vol_I_ch02_eq22}
\end{equation}
%--------------------------------------------------------------------------------
where ${\mathcal Q}$  is the specific external thermal power supply.

Inserting (\ref{eq:vol_I_ch02_eq21}), (\ref{eq:vol_I_ch02_eq21a}) 
and (\ref{eq:vol_I_ch02_eq22})
into (\ref{eq:vol_I_ch02_eq18}), we may express the energy balance
equation as
%-----------------------------------------------------------------------------
\begin{equation}
  \vint{\left[ \rho \dot{e} - \stress : \dot{\strain} - \microstress \cdot \gdot
   - {\mathcal Q} \right]} = 0
   \label{eq:vol_I_ch02_eq24}
\end{equation}
%--------------------------------------------------------------------------------
Localizing this result, i.e. the integrand must vanish
identically, we obtain theorem \ref{thm:local_energy_eq}.
\end{proof}

\begin{remark} In a locally {\em isometric} process $({\mathcal W}=0)$,
the net heat input is converted entirely into internal energy,
whereas in a locally {\em adiabatic} process $({\mathcal Q}=0)$, the
internal energy gained is the work done by the stresses. \textbf{REINTERPRET!!}
\end{remark}


\subsection{Entropy inequality -- Dissipation inequalities}
In order to complete the characterization of a thermodynamic system,
we shall also introduce the {\em entropy} $\entropy$. 
How the entropy can change due to the supply of heat power is governed by the
second law of thermodynamics, which formally takes the same form as in the 
standard situation without microforces present.

We introduce the {\em entropy function}
$\entropy$
%-----------------------------------------------------------------------------
\begin{equation}
  \entropy = \vint{\rho s}
     \label{eq:vol_I_ch02_eq28}
\end{equation}
%--------------------------------------------------------------------------------
where $s$ is the entropy density (entropy per unit mass). Associated with 
the existence of $\entropy$, we define ${\mathcal
R}_{\mathrm{ext}}$ as the rate of {\em input of entropy} from the
exterior into $\mathcal B$. The standard expression is
%-----------------------------------------------------------------------------
\begin{equation}
  \roentropy \defeq
  \vint{ \frac{r}{\theta} } +
  \sint{ \frac{q_\subn}{\theta} }
  %\label{eq:203.22}
     \label{eq:vol_I_ch02_eq31}
\end{equation}
%--------------------------------------------------------------------------------
The global format of the 2nd law (=axiom) of thermodynamics
 is {\em defined} as the inequality
%-----------------------------------------------------------------------------
\begin{equation}
  \dot{\entropy} - \roentropy \geq 0
%\label{eq:203.23}
     \label{eq:vol_I_ch02_eq30}
\end{equation}
%-----------------------------------------------------------------------------
Upon transforming the surface integral in (\ref{eq:vol_I_ch02_eq31})
into a  volume integral, we first obtain the representation
%--------------------------------------------------------------------------------
\begin{equation}
    \roentropy = v\int{ \mathcal{R} } \quad \mathrm{with} \quad
  {\mathcal R} \defeq
  \frac{\mathcal{Q} + \bm{h} \cdot \grad [\ln\theta]}{\theta}
%  \label{eq:203.a1}
 \label{eq:vol_I_ch02_eq32}
\end{equation}
%-----------------------------------------------------------------------------------
Inserting (\ref{eq:vol_I_ch02_eq28}) and (\ref{eq:vol_I_ch02_eq32})
into (\ref{eq:vol_I_ch02_eq30}), we may express the entropy inequality as
%--------------------------------------------------------------------------------
\begin{equation}
 \label{eq:vol_I_ch02_eq33}
  \vint{ [\rho \dot{s} - \mathcal{R}] } \geq 0 \quad \mbox{or} \quad
  \vint{ [\theta\rho \dot{s} - \mathcal{Q} - \bm{h} \cdot \grad[\ln\theta]] } \geq 0
\end{equation}
%-----------------------------------------------------------------------------------
which is commonly known as the global version of 
the {\em Clausius-Duhem-Inequality (CDI)}.

\begin{remark} The result in (\ref{eq:vol_I_ch02_eq33}) may be localized 
(in standard fashion) in the
sense that the integrand is non-negative; however, 
this is never exploited henceforth.
\end{remark}
%%--------------------------------------------------------------------------------
%\begin{equation}
%  \theta\rho \dot{s} - \mathcal{Q} - {\bm h}\cdot\grad[\ln\theta]
%  %\label{eq:203.a3}
% \label{eq:vol_I_ch02_eq34}
%\end{equation}
%%-----------------------------------------------------------------------------------

Alternatively, (\ref{eq:vol_I_ch02_eq33})$_2$ may be rewritten as the 
{\em dissipation inequality}
%----------------------------------------------------------------------------------
\begin{equation}
  \mathcal{D} = \mechdiss + \thermdiss \geq 0
 \label{eq:vol_I_ch02_eq35}
\end{equation}
%----------------------------------------------------------------------------------
where
%-----------------------------------------------------------------------------------
\begin{subequations}\label{eq:vol_I_ch02_eq36}
\begin{align}
  \mechdiss & \defeq \vint{[\rho\theta\dot{s} - \mathcal{Q}]} ,
\label{eq:vol_I_ch02_eq36a} \\
  \thermdiss & \defeq 
    \vint{ [-{\bm h}       \cdot \grad [\ln\theta]]}
  = \vint{ [-\frac{{\bm h} \cdot \grad \theta}{\theta}] }
\label{eq:vol_I_ch02_eq36b}
\end{align}
\end{subequations}
%------------------------------------------------------------------------------------
are the {\em mechanical} and {\em thermal} part, respectively, of the total 
dissipation. Upon introducing the local
format of the energy equation into (\ref{eq:vol_I_ch02_eq36a}), we obtain 
the alternative expression
%-----------------------------------------------------------------------------------
\begin{equation}
  \mechdiss = \vint{ \left[-\rho \dot{e} + \rho \theta \dot{s} +
  \stress: \dot{\strain} + \microstress \cdot \gdot \right]}
 \label{eq:vol_I_ch02_eq37}
\end{equation}
%------------------------------------------------------------------------------------
It is common to impose separately the (sufficient but not always necessary) conditions
%-----------------------------------------------------------------------------------
\begin{subequations}
\label{eq:vol_I_ch02_eq38}
\begin{align}
  \mechdiss & \geq 0,
\label{eq:vol_I_ch02_eq38a} \\
  \thermdiss & \geq 0
 \label{eq:vol_I_ch02_eq38b}
\end{align}
\end{subequations}
%------------------------------------------------------------------------------------
which are known as the {\em Clausius-Planck-Inequality (CPI)} and the 
{\em Fourier-Inequality (FI)}, respectively. This approach is taken subsequently.

\section{Constitutive framework - Isothermal conditions}

\subsection{Canonical format - 2-field format}

We first consider isothermal conditions, whereby the (absolute) temperature
$\theta=\theta_0$ serves only as a parameter in the constitutive model. 
We may the directly introduce the mass-specific free energy density 
$\psi\defeq e-\theta_0 s$ with the parameterization $\psi(\strain, k,\g)$.
From (\ref{eq:vol_I_ch02_eq37}), we then obtain
%----------------------------------------------------------------------------
\begin{eqnarray}
    \thermdiss(\udot, \dot{k})
    &=& \vint{ \left[-\rho \dot{\psi} + \stress : \dot{\strain} + 
    \microstress \cdot \gdot \right]} \nonumber \\
    &=&
    \vint{ \left[ \left[\stress -\rho \pfrac{\psi}{\strain} \right] : \strain [\udot] +
                  \left[-\rho \pfrac{\psi}{k} \right] \dot{k} +
                  \left[\microstress - \rho \pfrac{\psi}{\g} \right] \cdot \g[\dot{k}] 
            \right]}
    \nonumber \\
    &\geq& 0
    \quad \forall , (\udot ,\dot{k})
\label{eq:8-1}
\end{eqnarray}
that must hold for any given thermodynamic process defined by given fields 
$\udot,\dot{k}$. Note that these fields (in space-time) are treated as 
independent in this context.

The constitutive relations for the Standard Dissipative material with 
gradients under isothermal conditions are established as follows:
\begin{itemize}
  \item Introduce the mass-specific dissipation potential function 
    $\phi(\dot{\strain},\dot{k},\gdot)$, which (i) is convex
    \footnote{A function $F(\ul{x}):\;\bbR^{\mathrm{n}}\rightarrow\bbR$ is convex
    iff, for any pair $\ul{x}_1, \ul{x}_2$, the following inequality holds: 
    \[F(\alpha\ul{x}_1+[1-\alpha]\ul{x}_2)\leq\alpha F(\ul{x}_1)+[1-\alpha]F(\ul{x}_2)\].
    For a smooth convex function $F(\ul{x}):\;\bbR^{\mathrm{n}}\rightarrow\bbR$ 
    the following result holds: 
    \[ F(\ul{x}_2)- F(\ul{x}_1)\geq
    \left[\pfrac{F}{\ul{x}}(\ul{x}_1)\right]^{\mathrm{T}}
    [\ul{x}_2-\ul{x}_1]\quad\forall\ul{x}_1, \ul{x}_2\in\bbR^{\mathrm{n}}\]
    } 
    and (ii) satisfies the condition $\phi(\bm{0},0,\bm{0})=0$. 
    The corresponding global dissipation functional is

    $$ \Phi(\udot,\dot{k}) \defeq \vint{ \rho\phi(\strain[\udot],\dot{k},\g[\dot{k}]) } $$

  \item Introduce the global constitutive potential $\chi(\udot,\dot{k})\defeq 
    {\mathcal D}_{\mathrm{mech}}(\udot,\dot{k})-\Phi(\udot,\dot{k})$, and 
    the constitutive assumption that $\chi$ has a saddle point in the sense that

    $$ (\udot,\dot{k}) = \arg\left[\min_{\udot'}\max_{\dot{k}'} \chi(\udot',\dot{k}') \right]$$

    That the potential $\chi$ is stationary in the space of $(\udot,\dot{k})$
     for an actual thermodynamic process introduces a constraint on the relation 
     between these fields in space-time.
\end{itemize}

The directional (partial) derivatives of $\chi(\udot,\dot{k})$ for 
variations $\delta\udot$ of $\udot$ and $\delta\dot{k}$ of 
$\dot{k}$ are denoted $\chi'_{\dot{u}}(\udot,\dot{k};\delta\udot)$ 
and $\chi'_{\dot{k}}(\udot,\dot{k};\delta\dot{k})$, respectively. 
The stationarity condition corresponding to the saddle-point property then becomes
%----------------------------------------------------------------------------
\begin{subequations}\label{eq:8-2}
    \begin{align}
    \chi'_{\dot{u}}(\udot,\dot{k};\delta\udot)
    =&
    \vint{ \left[\stress - \rho\pfrac{\psi}{\strain}\right] \colon \strain[\delta\udot] } -
     \vint{ \left[\rho\pfrac{\phi}{\dot{\strain}}\right] \colon \strain[\delta\udot] }
    \nonumber \\
    =&
    \int_{\mathcal{B}} \left[\stress - \rho\pfrac{\psi}{\strain} -  
    \rho\pfrac{\phi}{\dot{\strain}} \right] \colon \strain[\delta\udot]  = 0,
    \quad \forall \delta\udot\in ???
\label{eq:8-2a}\\
    \chi'_{\dot{k}}(\udot,\dot{k};\delta\dot{k})
    =&
    \vint{ \left[-\rho\pfrac{\psi}{k} \delta\dot{k} +
    \left[\microstress - \rho\pfrac{\psi}{\g}\right] \cdot \g[\delta\dot{k}]\right] } -
    \vint{ \left[\rho\pfrac{\phi}{\dot{k}} \delta\dot{k} + 
    \rho\pfrac{\phi}{\gdot} \cdot \g[\delta\dot{k}]\right] }
    \nonumber \\
    =&
     -\int_{\mathcal{B}} \left[\rho\pfrac{\psi}{k} + \rho\pfrac{\phi}{\dot{k}} +
     \left[\microstress - \rho\pfrac{\psi}{\g} - \rho\pfrac{\phi}{\gdot}\right]
      \cdot \grad\right]\delta\dot{k} \id V
    \nonumber \\
     &+ \int_{\partial\mathcal{B}} \left[\microstress - \rho\pfrac{\psi}{\g} - 
     \rho\pfrac{\phi}\gdot\right] \cdot \normal\, \delta\dot{k}\id S = 0
    \quad \forall \delta\dot{k}\in ???
\label{eq:8-2b}
    \end{align}
\end{subequations}
%----------------------------------------------------------------------------
Upon localizing the result in (\ref{eq:8-2}), we obtain the constitutive identities
%----------------------------------------------------------------------------
\begin{subequations}\label{eq:8-203}
    \begin{align}
    \stress - \rho \pfrac{\psi}{\strain} -  
    \rho \pfrac{\phi}{\dot{\strain}} &= \bm{0} \quad
    \mbox{in } \mathcal{B}
\label{eq:8-203a}\\
    \rho\pfrac{\psi}{k} + \rho\pfrac{\phi}{\dot{k}} +
    \microstress\cdot \grad - \left[\rho\pfrac{\psi}{\g} + 
    \rho\pfrac{\phi}{\gdot}\right] \cdot \grad &= 0 \quad
    \mbox{in }  \mathcal{B}
\label{eq:8-203b} \\
    \microstress\cdot \normal - \left[\rho\pfrac{\psi}{\g} + 
    \rho\pfrac{\phi}{\gdot}\right] \cdot \normal &= 0 \quad
    \mbox{on }  \partial\mathcal{B}
\label{eq:8-203c}
    \end{align}
\end{subequations}
%----------------------------------------------------------------------------
Upon introducing the micro-stress variables
%----------------------------------------------------------------------------
\begin{subequations}\label{eq:8-105}
    \begin{align}
    \kappa &\defeq \rho\pfrac{\psi}{k} + \rho\pfrac{\phi}{\dot{k}}
\label{eq:8-105a}\\
    \bm{\xi} &\defeq \rho\pfrac{\psi}{\g} + \rho\pfrac{\phi}{\gdot}
\label{eq:8-105b}
    \end{align}
\end{subequations}
%----------------------------------------------------------------------------
we may abbreviate the system (\ref{eq:8-203b}, \ref{eq:8-203c}) as
%----------------------------------------------------------------------------
\begin{subequations}\label{eq:8-206}
    \begin{align}
    \kappa + \microstress\cdot\grad - \bm{\xi}\cdot\grad &= 0 \quad
    \mbox{in }  \mathcal{B}
\label{eq:8-206a} \\
    \microstress\cdot\normal -  \bm{\xi}\cdot\normal &= 0 \quad
    \mbox{on }  \partial\mathcal{B}
\label{eq:8-206b}
    \end{align}
\end{subequations}
%----------------------------------------------------------------------------
It is possible to eliminate the micro-stress $\microstress$ from the constitutive 
equations by combining (\ref{eq:8-206a}) with the equation for micro-equilibrium
 in (\ref{eq:vol_I_ch02_eq7a}). As a result, (\ref{eq:8-206a}) is replaced by 
 the equation
%----------------------------------------------------------------------------
\begin{equation}
    \kappa - \bm{\xi}\cdot\grad = b^{\mathrm{\mu}} \quad
    \mbox{in }  \mathcal{B}
\label{eq:8-207}
\end{equation}
%----------------------------------------------------------------------------
Moreover, from (\ref{eq:8-206b}), we note that the micro-traction on 
$\mathcal{B}$ can be expressed in terms of $\bm{\xi}$ as 
$t_{\mathrm{n}}^{\mathrm{\mu}}(\defeq\microstress\cdot\normal) 
=\bm{\xi}\cdot\normal$; hence, we conclude that the variable $\bm{\xi}$
 plays the role of a ``shifted'' micro-stress.

In the literature, e.g. {\scshape Biot} \cite{Biot1965}, {\scshape Nguyen} 
\cite{Nguyen2000}, the identity in (\ref{eq:8-203b}), or (\ref{eq:8-206a}),
is known as Biot's equation. This is a constitutive evolution equation,
which can be interpreted as a ``derived'' micro-equilibrium equation for
$\bm{\xi}$ of the Helmholtz type, whereby the constitutive relation for 
$\bm{\xi}$ was given in (\ref{eq:8-105b}). It plays a role that is simila
r to the standard equilibrium equation, whereby the constitutive relation
for $\stress$ is given directly in (\ref{eq:8-203a}). In conclusion,
upon combining these equations, we are in the position to solve for 
the fields $\bm{u}(\bm{x},t)$ and $k(\bm{x},t)$ in space-time for any
given body subjected to the appropriate loading and boundary conditions.


In order to comply with notation used in the literature, see e.g. 
{\scshape Gurtin} \cite{Gurtin2000}, we introduce the decomposition 
into ``energetic'' (superscript en) and ``dissipative'' (superscript di)
 parts as follows:
%----------------------------------------------------------------------------
\begin{subequations}\label{eq:8-409}
    \begin{align}
    \stress &= \stress^\en  + \stress^\di 
\label{eq:8-409a}\\
    \kappa &= \kappa^\en  + \kappa^\di 
\label{eq:8-409b}\\
    \bm{\xi} &= \bm{\xi}^\en  + \bm{\xi}^\di 
\label{eq:8-409c}
    \end{align}
\end{subequations}
%----------------------------------------------------------------------------
where
%----------------------------------------------------------------------------
\begin{subequations}\label{eq:8-209}
    \begin{align}
    \stress^\en (\strain,k,\g) &\defeq \rho\pfrac{\psi}{\strain}(\strain,k,\g), \quad
    \stress^\di (\dot{\strain},\dot{k},\gdot) \defeq \rho\pfrac{\phi}{\dot{\strain}}
    (\dot{\strain},\dot{k},\gdot)
\label{eq:8-209a}\\
    \kappa^\en (\strain,k,\g) &\defeq \rho\pfrac{\psi}{k}(\strain,k,\g), \quad
    \kappa^\di (\dot{\strain},\dot{k},\gdot) \defeq \rho\pfrac{\phi}{\dot{k}}
    (\dot{\strain},\dot{k},\gdot)
\label{eq:8-209b}\\
    \bm{\xi}^\en (\strain,k,\g) &\defeq \rho\pfrac{\psi}{\g}(\strain,k,\g), \quad
    \bm{\xi}^\di (\dot{\strain},\dot{k},\gdot) \defeq \rho\pfrac{\phi}{\gdot}
    (\dot{\strain},\dot{k},\gdot)
\label{eq:8-209c}
    \end{align}
\end{subequations}
%----------------------------------------------------------------------------
In conclusion, we may express (\ref{eq:8-207}) as
%----------------------------------------------------------------------------------------------------------------------
\begin{equation}
    \kappa^\en (\strain,k,\g) + \kappa^\di (\dot{\strain},\dot{k},\gdot) -
    \left[\bm{\xi}^\en (\strain,k,\g) + \bm{\xi}^\di (\dot{\strain},\dot{k},\gdot)\right]
    \cdot\grad= b^{\mathrm{\mu}}
\label{eq:8-404} 
\end{equation}
%----------------------------------------------------------------------------------------------------------------------

It remains to check that the CDI is satisfied for the \emph{actual} t
hermodynamic process defined by $(\udot,\dot{k})$ that satisfy the
 constitutive constraint equations (\ref{eq:8-206}). From (\ref{eq:8-1})
  we obtain
%----------------------------------------------------------------------------
\begin{eqnarray}
    {\mathcal D}_{\mathrm{mech}}(\udot,\dot{k})
    &=&
    \vint{ \left[\left[\stress -\rho\pfrac{\psi}{\strain}\right] 
    \colon \strain[\udot] + \left[- \rho\pfrac{\psi}{k}\right] \dot{k}
     + \left[\microstress -\rho\pfrac{\psi}{\g}\right] \cdot \g[\dot{k}] \right] }
    \nonumber \\
    &=& (\ref{eq:8-209}) =
    \vint{ \left[\left[\stress -\stress^\en \right] \colon \strain[\udot] +
     \left[-\kappa^\en \right] \dot{k} + \left[\microstress - \bm{\xi}^\en \right]
      \cdot \g[\dot{k}] \right] }
    \nonumber \\
    &=&
    \vint{ \left[\stress^\di \colon \strain[\udot] + \kappa^\di \dot{k} - 
    \kappa\dot{k} + \bm{\xi}^\di \cdot\g[\dot{k}] + \left[\microstress -
    \bm{\xi}\right]\cdot\g[\dot{k}] \right] }
    \nonumber \\
    &=&
    \vint{ \left[\stress^\di \colon \strain[\udot] + \kappa^\di  \dot{k} + 
    \bm{\xi}^\di  \cdot \g[\dot{k}] \right] }
    \nonumber \\
    &&- \vint{ \underbrace{\left[\kappa + \left[\microstress - \bm{\xi}\right] 
    \cdot \grad\right]}_{=0 \mbox{ from (\ref{eq:8-206a})}}\dot{k} }
    + \sint{ \underbrace{\left[\microstress - \bm{\xi}\right]
    \cdot\normal}_{=0 \mbox{ from (\ref{eq:8-206b})}} \dot{k} }
    \nonumber \\
    &=&
    \vint{ \underbrace{\left[\rho\pfrac{\phi}{\dot{\strain}}\colon 
    \strain[\udot]+ \rho\pfrac{\phi}{\dot{k}} \dot{k} + \rho\pfrac{\phi}{\gdot} 
    \cdot \g[\dot{k}] \right]}_{\geq 0} }
    \geq 0
    \quad \forall \udot,\dot{k}
\label{eq:8-902}
\end{eqnarray}
%----------------------------------------------------------------------------
That, indeed, ${\mathcal D}_{\mathrm{mech}}\geq 0$ follows directly 
from the properties of $\phi$ as given above.

\textbf{Proof}: Consider $X(\ul{x})$ convex and $X(\ul{0})=0$. From 
convexity follows that, for any given $\ul{x}$,
%----------------------------------------------------------------------------
\begin{equation}
    0 \leq X(\ul{x}) - \underbrace{X(\ul{0})}_{=0} 
    \leq X'(\ul{x})^{\mathrm{T}}[\ul{x} -\ul{0}] 
    \quad \Rightarrow \quad X'(\ul{x})^{\mathrm{T}}\ul{x}\geq 0
\label{eq:8-202}
\end{equation}
%----------------------------------------------------------------------------
Now, setting $X=\phi$ and $\ul{x}=(\dot{\strain},\dot{k},\gdot)$, we directly 
obtain the inequality in (\ref{eq:8-902}). $\Box$

\


\section{Constitutive framework - Non-isothermal conditions}

\subsection{Canonical format - 3-field format}

Next, we consider the general situation of non-isothermal conditions,
 whereby the (absolute) temperature $\theta$ (or the entropy $s$) is 
 included as a thermodynamic variable in the constitutive model. 
 The basic parameterization of the mass-specific internal energy density
  is $e(\strain,s,k,\g)$. From (\ref{eq:vol_I_ch02_eq37}), we now obtain
%----------------------------------------------------------------------------
\begin{eqnarray}
    {\mathcal D}_{\mathrm{mech}}(\udot,\dot{s},\dot{k})
    &=&
    \vint{ \left[-\rho\dot{e} + \rho\theta\dot{s} + 
    \stress\colon\dot{\strain} + \microstress \cdot \dot{\bm g}\right] }
    \nonumber \\
    &=&
    \int_{\mathcal{B}} \left[ \left[\stress -\rho\pfrac{e}{\strain}\right]
     \colon \strain[\udot]+ \left[\rho\theta - \rho\pfrac{e}{s}  \right] 
     \dot{s}+ \left[-\rho\pfrac{e}{k}\right] \dot{k} + \left[\microstress -
     \rho\pfrac{e}{\g}\right] \cdot \g[\dot{k}] \right]\id V
    \nonumber \\
    &\geq&  0
    \quad \forall \udot,\dot{s},\dot{k}
\label{eq:8-201}
\end{eqnarray}
%----------------------------------------------------------------------------
that must hold for any given thermodynamic process defined by given fields
 $\udot,\dot{s},\dot{k}$. Note that these fields (in space-time) treated 
 as independent in this context.

The constitutive relations for the Standard Dissipative material with 
gradients under non-isothermal conditions are established as follows:
\begin{itemize}
  \item Introduce the mass-specific dissipation potential function 
  $\phi(\dot{\strain},\dot{k},\gdot)$, which (i) is convex and (ii) 
  satisfies the condition $\phi(\bm{0},0,\bm{0})=0$. The corresponding
  global dissipation functional is

  $$ \Phi(\udot,\dot{k}) \defeq \vint{ \rho\phi(\strain[\udot],\dot{k},\g[\dot{k}]) } $$

  \item Introduce the global constitutive potential 
  $\chi(\udot,\dot{s},\dot{k})\defeq {\mathcal D}_{\mathrm{mech}}(\udot,
  \dot{s},\dot{k})-\Phi(\udot,\dot{k})$, and the constitutive assumption 
  that $\chi$ has a saddle point in the sense that

  $$ (\udot,\dot{s},\dot{k}) = \arg\left[\min_{\udot'}\min_{\dot{s}'}\max_{\dot{k}'}
  \chi(\udot',\dot{s}',\dot{k}') \right]$$

That the potential $\chi$ is stationary in the space of $(\udot,\dot{s},\dot{k})$ 
for an actual thermodynamic process introduces a constraint on the relation between 
these fields in space-time.
\end{itemize}

The stationarity condition corresponding to the saddle-point property then becomes
%----------------------------------------------------------------------------
\begin{subequations}\label{eq:8-302}
    \begin{align}
    \chi'_{\dot{u}}(\udot,\dot{s},\dot{k};\delta\udot)
    =&
    \vint{ \left[\stress - \rho\pfrac{e}{\strain}\right] \colon \strain[\delta\udot] } - 
    \vint{ \left[\rho\pfrac{\phi}{\dot{\strain}}\right] \colon \strain[\delta\udot] }
    \nonumber \\
    =&
    \int_{\mathcal{B}} \left[\stress - \rho\pfrac{e}{\strain} -  
    \rho\pfrac{\phi}{\dot{\strain}} \right] \colon \strain[\delta\udot]  = 0,
    \quad \forall \delta\udot\in ???
\label{eq:8-302a}\\
    \chi'_{\dot{u}}(\udot,\dot{s},\dot{k};\delta\dot{s})
    =&
    \vint{ \left[\rho\theta - \rho\pfrac{e}{s}\right] \delta\dot{s} }
    \nonumber \\
    =&
    0,
    \quad \forall \delta\dot{s}\in ???
\label{eq:8-302b}\\
    \chi'_{\dot{k}}(\udot,\dot{s},\dot{k};\delta\dot{k})
    =&
    \vint{ \left[-\rho\pfrac{e}{k} \delta\dot{k} + \left[\microstress - 
    \rho\pfrac{e}{\g}\right] \cdot \g[\delta\dot{k}]\right] } -
    \vint{ \left[\rho\pfrac{\phi}{\dot{k}} \delta\dot{k} + 
    \rho\pfrac{\phi}{\gdot} \cdot \g[\delta\dot{k}]\right] }
    \nonumber \\
    =&
     -\int_{\mathcal{B}} \left[\rho\pfrac{e}{k} + 
     \rho\pfrac{\phi}{\dot{k}} +
     \left[\microstress - \rho\pfrac{e}{\g} - 
     \rho\pfrac{\phi}{\gdot}\right] \cdot \grad\right]\delta\dot{k} \id V
    \nonumber \\
     &+ \int_{\partial\mathcal{B}} \left[\microstress - \rho\pfrac{e}{\g} - 
     \rho\pfrac{\phi}{\gdot}\right] \cdot \normal\, \delta\dot{k}\id S = 0
    \quad \forall \delta\dot{k}\in ???
\label{eq:8-302c}
    \end{align}
\end{subequations}
%----------------------------------------------------------------------------
Upon localizing the result in (\ref{eq:8-302}), we obtain 
the constitutive identities
%----------------------------------------------------------------------------
\begin{subequations}\label{eq:8-303}
    \begin{align}
    \stress - \rho \pfrac{e}{\strain} -  
    \rho \pfrac{\phi}{\dot{\strain}} &= \bm{0} \quad
    \mbox{in } \mathcal{B}
\label{eq:8-303a}\\
    \theta - \pfrac{e}{s} &= \bm{0} \quad
    \mbox{in } \mathcal{B}
\label{eq:8-303b}\\
    \rho\pfrac{e}{k} + \rho\pfrac{\phi}{\dot{k}} +
    \microstress \cdot \grad - \left[\rho\pfrac{e}{\g} + 
    \rho\pfrac{\phi}{\gdot}\right] \cdot \grad &= 0 \quad
    \mbox{in }  \mathcal{B}
\label{eq:8-303c} \\
    \microstress \cdot \normal - \left[\rho\pfrac{e}{\g} + 
    \rho\pfrac{\phi}{\gdot}\right] \cdot \normal &= 0 \quad
    \mbox{on }  \partial\mathcal{B}
\label{eq:8-303d}
    \end{align}
\end{subequations}
%----------------------------------------------------------------------------
From (\ref{eq:8-303b}) appears that the tempertaure $\theta$ is purely energetic. 
It is then convenient to introduce the mass-specific free energy density $\psi$
 via the Legendre transformation
%---------------------------------------------------------------------------------------------------
\begin{equation}
    \psi(\strain,\theta,k,\g) = \inf_{\hat{s}}
    \left[ e(\strain,\hat{s},k,\g) - \theta \hat{s} \right]
\label{eq:8-211}
\end{equation}
%---------------------------------------------------------------------------------------------------
Upon evaluating the inf, we establish the (stationarity) condition
%---------------------------------------------------------------------------------------------------
\begin{equation}
    \pfrac{e}{s}(\strain,s,k,\g) = \theta
\label{eq:8-212}
\end{equation}
%---------------------------------------------------------------------------------------------------
which is precisely the constitutive relation for $\theta$ already obtained 
in (\ref{eq:8-303b}). Assuming that $\theta$ is monotonic in $s$ (for fixed
$\strain,k$ and $\g$, we may solve for $s=\bar{s}(\strain,\theta,k,\g)$ 
from (\ref{eq:8-212}) and insert into the expression for $\psi$ in (\ref{eq:8-211})
to obtain
%---------------------------------------------------------------------------------------------------
\begin{equation}
    \psi(\strain,\theta,k,\g) = e(\strain,\bar{s}(\strain,\theta,k,\g),k,\g) - 
    \theta \bar{s}(\strain,\theta,k,\g)
\label{eq:8-213}
\end{equation}
%---------------------------------------------------------------------------------------------------
Hence, the conditions are
%----------------------------------------------------------------------------
\begin{subequations}\label{eq:8-214}
    \begin{align}
    \pfrac{\psi}{\strain}
    &=
    \pfrac{e}{\strain}|_{s} + \left[\underbrace{\pfrac{e}{\bar{s}} - 
    \theta}_{=0} \right] \pfrac{\bar{s}}{\strain} = \pfrac{e}{\strain}
\label{eq:8-214a}\\
    \pfrac{\psi}{\theta}
    &=
    \left[\underbrace{\pfrac{e}{\bar{s}} - \theta}_{=0} \right] 
    \pfrac{\bar{s}}{\theta} -s  = -s
\label{eq:8-214b} \\
    \pfrac{\psi}{k}
    &=
    \pfrac{e}{k}|_{s} + \left[\underbrace{\pfrac{e}{\bar{s}} -
     \theta}_{=0} \right] \pfrac{\bar{s}}{k} = \pfrac{e}{k}
\label{eq:8-214c} \\
    \pfrac{\psi}{\g}
    &=
    \pfrac{e}{\g}|_{s} + \left[\underbrace{\pfrac{e}{\bar{s}} - 
    \theta}_{=0} \right] \pfrac{\bar{s}}{\g} = \pfrac{e}{\g}
\label{eq:8-214c}
    \end{align}
\end{subequations}
%----------------------------------------------------------------------------
As a result, the constitutive identities (\ref{eq:8-303}) are replaced by
%----------------------------------------------------------------------------
\begin{subequations}\label{eq:8-203}
    \begin{align}
    \stress - \rho \pfrac{\psi}{\strain} -  
    \rho \pfrac{\phi}{\dot{\strain}} &= \bm{0} \quad
    \mbox{in } \mathcal{B}
\label{eq:8-203a}\\
    s + \pfrac{\psi}{\theta} &= \bm{0} \quad
    \mbox{in } \mathcal{B}
\label{eq:8-203b}\\
    \rho\pfrac{\psi}{k} + \rho\pfrac{\phi}{\dot{k}} +
    \microstress\cdot \grad - \left[\rho\pfrac{\psi}{\g} + 
    \rho\pfrac{\phi}{\gdot}\right] \cdot \grad &= 0 \quad
    \mbox{in }  \mathcal{B}
\label{eq:8-203c} \\
    \microstress\cdot \normal - \left[\rho\pfrac{\psi}{\g} + 
  \rho\pfrac{\phi}{\gdot}\right] \cdot \normal &= 0 \quad
    \mbox{on }  \partial\mathcal{B}
\label{eq:8-203d}
    \end{align}
\end{subequations}
%----------------------------------------------------------------------------
We may now put forward the same arguments as for the isothermal situation to 
conclude that the relevant Biot equation is the same as in (\ref{eq:8-207}), i.e.
%----------------------------------------------------------------------------
\begin{equation}
    \kappa - \bm{\xi}\cdot\grad = b^{\mathrm{\mu}} \quad
    \mbox{in }  \mathcal{B}
\label{eq:8-207a}
\end{equation}
%----------------------------------------------------------------------------
whereby the variable $\bm{\xi}$ plays the role of a ``shifted'' micro-stress.

As to the decomposition into ``energetic'' (superscript en) and ``dissipative'' 
(superscript di) parts, we note that $s$ is purely energetic. In summary,
%----------------------------------------------------------------------------
\begin{subequations}\label{eq:8-419}
    \begin{align}
    \stress &= \stress^\en  + \stress^\di 
\label{eq:8-419a}\\
    s &= s^\en 
\label{eq:8-419b}\\
    \kappa &= \kappa^\en  + \kappa^\di 
\label{eq:8-419c}\\
    \bm{\xi} &= \bm{\xi}^\en  + \bm{\xi}^\di 
\label{eq:8-419c}
    \end{align}
\end{subequations}
%----------------------------------------------------------------------------
where
%----------------------------------------------------------------------------
\begin{subequations}\label{eq:8-219}
    \begin{align}
    \stress^\en (\strain,\theta,k,\g) &\defeq \rho\pfrac{\psi}{\strain}(\strain,\theta,k,\g), \quad
    \stress^\di (\dot{\strain},\dot{k},\gdot) \defeq \rho\pfrac{\phi}{\dot{\strain}}(\dot{\strain},\dot{k},\gdot)
\label{eq:8-219a}\\
    s^\en (\strain,\theta,k,\g) &\defeq -\pfrac{\psi}{\theta}(\strain,\theta,k,\g)
\label{eq:8-219b}\\
    \kappa^\en (\strain,\theta,k,\g) &\defeq \rho\pfrac{\psi}{\ul{k}}(\strain,\theta,k,\g), \quad
    \kappa^\di (\dot{\strain},\dot{k},\gdot) \defeq \rho\pfrac{\phi}{\dot{\ul{k}}}(\dot{\strain},\dot{k},\gdot)
\label{eq:8-219c}\\
    \bm{\xi}^\en (\strain,\theta,k,\g) &\defeq \rho\pfrac{\psi}{\g}(\strain,\theta,k,\g), \quad
    \bm{\xi}^\di (\dot{\strain},\dot{k},\gdot) \defeq \rho\pfrac{\phi}{\gdot}(\dot{\strain},\dot{k},\gdot)
\label{eq:8-219d}
    \end{align}
\end{subequations}

Finally, that ${\mathcal D}_{\mathrm{mech}}\geq 0$ can be shown as for the isothermal situation.




















